\documentclass[10pt,twocolumn,letterpaper]{article}

\usepackage{cvpr}
\usepackage{times}
\usepackage{epsfig}
\usepackage{graphicx}
\usepackage{amsmath}
\usepackage{amssymb}

% Include other packages here, before hyperref.

% If you comment hyperref and then uncomment it, you should delete
% egpaper.aux before re-running latex.  (Or just hit 'q' on the first latex
% run, let it finish, and you should be clear).
\usepackage[breaklinks=true,bookmarks=false]{hyperref}

\cvprfinalcopy % *** Uncomment this line for the final submission

\def\cvprPaperID{****} % *** Enter the CVPR Paper ID here
\def\httilde{\mbox{\tt\raisebox{-.5ex}{\symbol{126}}}}

% Pages are numbered in submission mode, and unnumbered in camera-ready
%\ifcvprfinal\pagestyle{empty}\fi
\setcounter{page}{4321}
\begin{document}

%%%%%%%%% TITLE
\title{Few-Shot Logo Recognition}

\author{Oggero Paolo\\
s342937\\
address:\\
{\tt\small TODO: insert email address}
% For a paper whose authors are all at the same institution,
% omit the following lines up until the closing ``}''.
% Additional authors and addresses can be added with ``\and'',
% just like the second author.
% To save space, use either the email address or home page, not both
\and
Cancemi Alessia\\
s347156\\
address\\
{\tt\small TODO: insert email address}
\and
Mincigrucci Gabriele\\
s358987\\
address\\
{\tt\small TODO: insert email address}
}

\maketitle
%\thispagestyle{empty}

%%%%%%%%% ABSTRACT
\begin{abstract}
   Logo recognition in open-world scenarios is hampered by the long-tailed
   nature of the data. This work presents a Few-Shot Learning pipeline based
   on LogoDet-3K and Deep Metric Learning. Using a ResNet-50 optimized with
   Triplet and Contrastive Loss, we map logos into a 128-dimensional embedding space.
   Through a progressive freezing strategy, the model learns generalized representations
   that allow the retrieval of brands never seen during training. The results demonstrate
   that optimizing the latent space geometry significantly improves the mAP and F1-score
   compared to standard classification methods.
\end{abstract}

%%%%%%%%% BODY TEXT
\section{Introduction}

Accurate logo recognition has become a fundamental pillar of media analysis and
copyright protection. In uncontrolled environments, however, computer vision systems
must contend with market dynamics: new brands emerge every day with unique visual identities
that must be instantly identified. Traditional classification paradigms suffer from two
structural limitations: the need for enormous datasets for each class and the inability
to recognize categories not included in the training set. This phenomenon, known as 
long-tailed distribution, makes models rigid and expensive to update. Few-Shot Learning emerges as a necessary solution, shifting the focus from "mnemonic recognition" to
"morphological comparison." 
\begin{figure}[!h] % [t] forza l'immagine in alto (Top) logodet3k_with_red_dots
   \centering
   % Sostituisci 'nome_immagine' con il nome del file (senza estensione se vuoi)
   % width=\linewidth adatta l'immagine esattamente alla larghezza della colonna
   \includegraphics[width=\linewidth]{1immintro.pdf}
   
   \caption{Sample of images from LogoDet-3K}
   \label{fig:esempio_singolo} % Label per citarla nel testo
\end{figure}
In this work, we address this challenge by implementing a Deep
Metric Learning system. Instead of training the model to assign a label, we train it to "measure"
similarity. Using the LogoDet-3K dataset, we developed a pipeline that extracts deep features
using a ResNet-50 and projects them into a metric space where the Euclidean distance reflects the
semantic relatedness of the logos. This approach not only better manages data sparsity, but also
allows the system to operate on unseen classes without the need for retraining. The approach
presented here focuses on overcoming the limitations of traditional classification through
various technical solutions. The work first introduces an embedding-based architecture,
developed using a linear projection head that enables a standard CNN to extract metric features.
This structure allows for an in-depth comparative analysis of losses, evaluating how Triplet and
Contrastive Loss (in the Euclidean and Cosine variants) influence the topology of the latent space.
To support training, a dynamic transfer learning management system based on progressive layer
unlocking is implemented, ensuring an optimal balance between model stability and performance.
The work concludes with an Open-Set Evaluation conducted on brands never encountered during the
training phase, using retrieval metrics such as mAP to validate the system's actual generalization capability.

%-------------------------------------------------------------------------
\section{Data}
For this project, we used the LogoDet-3K dataset\cite{wang2022logodet}, which currently
 represents one of the largest and most complex benchmarks for logo recognition.

\subsection{Dataset Description}
LogoDet-3K comprises 3,000 logo categories, with approximately 200,000 manually
 annotated objects distributed across 158,652 images. The dataset is hierarchically 
 organized into nine supercategories (Food, Clothes, Necessities, Others, Electronics, 
 Transportation, Leisure, Sports, and Medical). The dataset is inherently long-tailed, we can see that in figure 2, 
 some classes have a very large number of samples, while others (especially in the Medical 
 or Sports categories) have very few instances. This distribution faithfully reflects the 
 challenges of real-world scenarios.
 \begin{figure}[!h] % [t] forza l'immagine in alto (Top) 
   \centering
   % Sostituisci 'nome_immagine' con il nome del file (senza estensione se vuoi)
   % width=\linewidth adatta l'immagine esattamente alla larghezza della colonna
   \includegraphics[width=\linewidth]{logodet3k_with_red_dots.pdf}
   
   \caption{Distribution of the LogoDet-3K dataset by macro-category}
   \label{fig:esempio_singolo} % Label per citarla nel testo
\end{figure}

\subsection{Splitting of the Dataset (Brand-Level Split)}
Unlike standard classification pipelines, where images are randomly shuffled and split,
our code implements the \texttt{getPathsSetsByBrand} function. This logic applies splitting at the
class (brand) level rather than at the individual image level:
\begin{itemize}
    \setlength{\itemsep}{0pt}
    \setlength{\parskip}{0pt}
    \setlength{\parsep}{0pt}
    \item Training Set (70\% of brands): The model learns the general morphological characteristics of logos.
    \item Validation Set (20\% of brands): Used to monitor the F1 score and convergence on brands never seen during training.
    \item Test Set (10\% of brands): Reserved for the final performance evaluation in Open-Set Retrieval mode.
\end{itemize}

\subsection{Data Preparation and Sampling}
To meet the requirements of the implemented loss functions, two specific Dataset classes have been created:
\begin{enumerate}
    \setlength{\itemsep}{0pt}
    \setlength{\parskip}{0pt}
    \setlength{\parsep}{0pt}
    \item \texttt{DatasetTriplet}: For each reference image (Anchor), randomly select one example of the same brand (Positive) and one of a different brand (Negative).
    \item \texttt{DatasetContrastive}: Generates image pairs with a 50\% probability of belonging to the same brand (label 1) or different brands (label 0).
\end{enumerate}

\subsection{Preprocessing and Data Augmentation}

The original images have heterogeneous resolutions. In the loading module, the data is normalized and transformed to improve the model's robustness:

\begin{itemize}
    \item \textbf{Resize and Normalization:} All images are resized to $224 \times 224$ pixels and normalized using the ImageNet mean and standard deviation, ensuring compatibility with the pre-trained weights of ResNet-50.
    
    \item \textbf{Augmentation:} During training, transformations are applied, including \texttt{RandomResizedCrop}, \texttt{RandomHorizontalFlip}, and \texttt{ColorJitter}. These techniques simulate the distortions typical of real-world logos (light variations, angled shots, etc.).
\end{itemize}


\section{Methods}
This section describes the system architecture and the optimization methodologies adopted to transform the logo recognition problem into a Deep Metric Learning task.

\subsection{Model Architecture: LogoResNet50}
The implemented architecture is based on the ResNet-50 backbone, chosen for its optimal balance between computational depth and feature extraction capability. The original model, pre-trained on ImageNet, was modified to adapt to the metric learning paradigm through two structural interventions:
\begin{itemize}
    \item \textbf{Classifier Replacement:} The final Fully Connected layer (originally designed for 1,000 classes) was removed and replaced with a Linear Embedding Head. This module projects the high-level features extracted by the network into a $128$-dimensional latent space.
    
    \item \textbf{Space Normalization:} The output embeddings are treated as geometric coordinates; the network learns to place similar logos in nearby regions of the vector space.
\end{itemize}


\subsection{Loss Functions}
To optimize the topology of the embedding space, three different training objectives were implemented and compared:

\begin{itemize}
    \item \textbf{Triplet Margin Loss}: This is the primary retrieval strategy. The function works on triplets (Anchor, Positive, Negative) and minimizes the distance between Anchor and Positive, while simultaneously maximizing the distance between Anchor and Negative with a configurable margin ($\alpha= 1.0 $).
    \item \textbf{Contrastive Loss (Euclidean)}: Optimized for image pairs, it penalizes the distance between similar pairs and forces dissimilar pairs to a distance greater than the set margin.    
    \item \textbf{Contrastive Loss (Cosine)}: A variant of the previous one that uses cosine similarity instead of Euclidean distance. This metric is particularly effective in high-dimensional spaces because it focuses on the orientation of the vectors rather than their magnitude.

\end{itemize}

\subsection{Training and Transfer Learning Strategy}
Training is managed by a dynamic configuration system (Config) that allows for reproducible experiments. The main phases include:
\begin{itemize}
   \item \textbf{Progressive Freezing}: To preserve pre-learned knowledge about ImageNet, the model starts training with frozen backbone convolutional blocks (freeze\_layers).
   \item \textbf{Dynamic Unfreezing}: Upon reaching a predetermined epoch (\texttt{unfreeze\_at\_epoch}$ = 5$), the code unfreezes layers, allowing for fine-tuning of logo-specific spatial features.
   \item \textbf{Optimization}: The Adam optimizer with a conservative learning rate $(1e-5)$ is used to ensure stable convergence and minimize oscillations in the Loss.
\end{itemize}

\subsection{Evaluation Metrics}
The system is evaluated not simply on accuracy, but on its ranking ability. The main metrics calculated in the validation loop are:
\begin{itemize}
   \item \textbf{F1-Score}: Dynamically calculated during training to monitor the balance between Precision and Recall.
   \item \textbf{Mean Average Precision (mAP)}: Used to measure retrieval quality, i.e., how effectively the model ranks correct logos at the top of search results.
\end{itemize}





%
%An example of a bad paper just asking to be rejected:
%\begin{quote}
%\begin{center}
%    An analysis of the frobnicatable foo filter.
%\end{center}
%
%   In this paper we present a performance analysis of our
%   previous paper [1], and show it to be inferior to all
%   previously known methods.  Why the previous paper was
%   accepted without this analysis is beyond me.
%
%   [1] Removed for blind review
%\end{quote}
%
%
%An example of an acceptable paper:
%
%\begin{quote}
%\begin{center}
%     An analysis of the frobnicatable foo filter.
%\end{center}
%
%   In this paper we present a performance analysis of the
%   paper of Smith \etal [1], and show it to be inferior to
%   all previously known methods.  Why the previous paper
%   was accepted without this analysis is beyond me.
%
%   [1] Smith, L and Jones, C. ``The frobnicatable foo
%   filter, a fundamental contribution to human knowledge''.
%   Nature 381(12), 1-213.
%\end{quote}
%
%If you are making a submission to another conference at the same time,
%which covers similar or overlapping material, you may need to refer to that
%submission in order to explain the differences, just as you would if you
%had previously published related work.  In such cases, include the
%anonymized parallel submission~\cite{Authors14} as additional material and
%cite it as
%\begin{quote}
%[1] Authors. ``The frobnicatable foo filter'', F\&G 2014 Submission ID 324,
%Supplied as additional material {\tt fg324.pdf}.
%\end{quote}
%
%Finally, you may feel you need to tell the reader that more details can be
%found elsewhere, and refer them to a technical report.  For conference
%submissions, the paper must stand on its own, and not {\em require} the
%reviewer to go to a techreport for further details.  Thus, you may say in
%the body of the paper ``further details may be found
%in~\cite{Authors14b}''.  Then submit the techreport as additional material.
%Again, you may not assume the reviewers will read this material.
%
%Sometimes your paper is about a problem which you tested using a tool which
%is widely known to be restricted to a single institution.  For example,
%let's say it's 1969, you have solved a key problem on the Apollo lander,
%and you believe that the CVPR70 audience would like to hear about your
%solution.  The work is a development of your celebrated 1968 paper entitled
%``Zero-g frobnication: How being the only people in the world with access to
%the Apollo lander source code makes us a wow at parties'', by Zeus \etal.
%
%You can handle this paper like any other.  Don't write ``We show how to
%improve our previous work [Anonymous, 1968].  This time we tested the
%algorithm on a lunar lander [name of lander removed for blind review]''.
%That would be silly, and would immediately identify the authors. Instead
%write the following:
%\begin{quotation}
%\noindent
%   We describe a system for zero-g frobnication.  This
%   system is new because it handles the following cases:
%   A, B.  Previous systems [Zeus et al. 1968] didn't
%   handle case B properly.  Ours handles it by including
%   a foo term in the bar integral.
%
%   ...
%
%   The proposed system was integrated with the Apollo
%   lunar lander, and went all the way to the moon, don't
%   you know.  It displayed the following behaviours
%   which show how well we solved cases A and B: ...
%\end{quotation}
%As you can see, the above text follows standard scientific convention,
%reads better than the first version, and does not explicitly name you as
%the authors.  A reviewer might think it likely that the new paper was
%written by Zeus \etal, but cannot make any decision based on that guess.
%He or she would have to be sure that no other authors could have been
%contracted to solve problem B.
%\medskip
%
%\noindent
%FAQ\medskip\\
%{\bf Q:} Are acknowledgements OK?\\
%{\bf A:} No.  Leave them for the final copy.\medskip\\
%{\bf Q:} How do I cite my results reported in open challenges?
%{\bf A:} To conform with the double blind review policy, you can report results of other challenge participants together with your results in your paper. For your results, however, you should not identify yourself and should not mention your participation in the challenge. Instead present your results referring to the method proposed in your paper and draw conclusions based on the experimental comparison to other results.\medskip\\
%
%
%
%\begin{figure}[t]
%\begin{center}
%\fbox{\rule{0pt}{2in} \rule{0.9\linewidth}{0pt}}
%   %\includegraphics[width=0.8\linewidth]{egfigure.eps}
%\end{center}
%   \caption{Example of caption.  It is set in Roman so that mathematics
%   (always set in Roman: $B \sin A = A \sin B$) may be included without an
%   ugly clash.}
%\label{fig:long}
%\label{fig:onecol}
%\end{figure}
%
%\subsection{Miscellaneous}
%
%\noindent
%Compare the following:\\
%\begin{tabular}{ll}
% \verb'$conf_a$' &  $conf_a$ \\
% \verb'$\mathit{conf}_a$' & $\mathit{conf}_a$
%\end{tabular}\\
%See The \TeX book, p165.
%
%The space after \eg, meaning ``for example'', should not be a
%sentence-ending space. So \eg is correct, {\em e.g.} is not.  The provided
%\verb'\eg' macro takes care of this.
%
%When citing a multi-author paper, you may save space by using ``et alia'',
%shortened to ``\etal'' (not ``{\em et.\ al.}'' as ``{\em et}'' is a complete word.)
%However, use it only when there are three or more authors.  Thus, the
%following is correct: ``
%   Frobnication has been trendy lately.
%   It was introduced by Alpher~\cite{Alpher02}, and subsequently developed by
%   Alpher and Fotheringham-Smythe~\cite{Alpher03}, and Alpher \etal~\cite{Alpher04}.''
%
%This is incorrect: ``... subsequently developed by Alpher \etal~\cite{Alpher03} ...''
%because reference~\cite{Alpher03} has just two authors.  If you use the
%\verb'\etal' macro provided, then you need not worry about double periods
%when used at the end of a sentence as in Alpher \etal.
%
%For this citation style, keep multiple citations in numerical (not
%chronological) order, so prefer \cite{Alpher03,Alpher02,Authors14} to
%\cite{Alpher02,Alpher03,Authors14}.
%
%
%\begin{figure*}
%\begin{center}
%\fbox{\rule{0pt}{2in} \rule{.9\linewidth}{0pt}}
%\end{center}
%   \caption{Example of a short caption, which should be centered.}
%\label{fig:short}
%\end{figure*}
%
%%------------------------------------------------------------------------
%\section{Formatting your paper}
%
%All text must be in a two-column format. The total allowable width of the
%text area is $6\frac78$ inches (17.5 cm) wide by $8\frac78$ inches (22.54
%cm) high. Columns are to be $3\frac14$ inches (8.25 cm) wide, with a
%$\frac{5}{16}$ inch (0.8 cm) space between them. The main title (on the
%first page) should begin 1.0 inch (2.54 cm) from the top edge of the
%page. The second and following pages should begin 1.0 inch (2.54 cm) from
%the top edge. On all pages, the bottom margin should be 1-1/8 inches (2.86
%cm) from the bottom edge of the page for $8.5 \times 11$-inch paper; for A4
%paper, approximately 1-5/8 inches (4.13 cm) from the bottom edge of the
%page.
%
%%-------------------------------------------------------------------------
%\subsection{Margins and page numbering}
%
%All printed material, including text, illustrations, and charts, must be kept
%within a print area 6-7/8 inches (17.5 cm) wide by 8-7/8 inches (22.54 cm)
%high.
%Page numbers should be in footer with page numbers, centered and .75
%inches from the bottom of the page and make it start at the correct page
%number rather than the 4321 in the example.  To do this fine the line (around
%line 23)
%\begin{verbatim}
%%\ifcvprfinal\pagestyle{empty}\fi
%\setcounter{page}{4321}
%\end{verbatim}
%where the number 4321 is your assigned starting page.
%
%Make sure the first page is numbered by commenting out the first page being
%empty on line 46
%\begin{verbatim}
%%\thispagestyle{empty}
%\end{verbatim}
%
%
%%-------------------------------------------------------------------------
%\subsection{Type-style and fonts}
%
%Wherever Times is specified, Times Roman may also be used. If neither is
%available on your word processor, please use the font closest in
%appearance to Times to which you have access.
%
%MAIN TITLE. Center the title 1-3/8 inches (3.49 cm) from the top edge of
%the first page. The title should be in Times 14-point, boldface type.
%Capitalize the first letter of nouns, pronouns, verbs, adjectives, and
%adverbs; do not capitalize articles, coordinate conjunctions, or
%prepositions (unless the title begins with such a word). Leave two blank
%lines after the title.
%
%AUTHOR NAME(s) and AFFILIATION(s) are to be centered beneath the title
%and printed in Times 12-point, non-boldface type. This information is to
%be followed by two blank lines.
%
%The ABSTRACT and MAIN TEXT are to be in a two-column format.
%
%MAIN TEXT. Type main text in 10-point Times, single-spaced. Do NOT use
%double-spacing. All paragraphs should be indented 1 pica (approx. 1/6
%inch or 0.422 cm). Make sure your text is fully justified---that is,
%flush left and flush right. Please do not place any additional blank
%lines between paragraphs.
%
%Figure and table captions should be 9-point Roman type as in
%Figures~\ref{fig:onecol} and~\ref{fig:short}.  Short captions should be centred.
%
%\noindent Callouts should be 9-point Helvetica, non-boldface type.
%Initially capitalize only the first word of section titles and first-,
%second-, and third-order headings.
%
%FIRST-ORDER HEADINGS. (For example, {\large \bf 1. Introduction})
%should be Times 12-point boldface, initially capitalized, flush left,
%with one blank line before, and one blank line after.
%
%SECOND-ORDER HEADINGS. (For example, { \bf 1.1. Database elements})
%should be Times 11-point boldface, initially capitalized, flush left,
%with one blank line before, and one after. If you require a third-order
%heading (we discourage it), use 10-point Times, boldface, initially
%capitalized, flush left, preceded by one blank line, followed by a period
%and your text on the same line.
%
%%-------------------------------------------------------------------------
%\subsection{Footnotes}
%
%Please use footnotes\footnote {This is what a footnote looks like.  It
%often distracts the reader from the main flow of the argument.} sparingly.
%Indeed, try to avoid footnotes altogether and include necessary peripheral
%observations in
%the text (within parentheses, if you prefer, as in this sentence).  If you
%wish to use a footnote, place it at the bottom of the column on the page on
%which it is referenced. Use Times 8-point type, single-spaced.
%
%
%%-------------------------------------------------------------------------
%\subsection{References}
%
%List and number all bibliographical references in 9-point Times,
%single-spaced, at the end of your paper. When referenced in the text,
%enclose the citation number in square brackets, for
%example~\cite{Authors14}.  Where appropriate, include the name(s) of
%editors of referenced books.
%
%\begin{table}
%\begin{center}
%\begin{tabular}{|l|c|}
%\hline
%Method & Frobnability \\
%\hline\hline
%Theirs & Frumpy \\
%Yours & Frobbly \\
%Ours & Makes one's heart Frob\\
%\hline
%\end{tabular}
%\end{center}
%\caption{Results.   Ours is better.}
%\end{table}
%
%%-------------------------------------------------------------------------
%\subsection{Illustrations, graphs, and photographs}
%
%All graphics should be centered.  Please ensure that any point you wish to
%make is resolvable in a printed copy of the paper.  Resize fonts in figures
%to match the font in the body text, and choose line widths which render
%effectively in print.  Many readers (and reviewers), even of an electronic
%copy, will choose to print your paper in order to read it.  You cannot
%insist that they do otherwise, and therefore must not assume that they can
%zoom in to see tiny details on a graphic.
%
%When placing figures in \LaTeX, it's almost always best to use
%\verb+\includegraphics+, and to specify the  figure width as a multiple of
%the line width as in the example below
%{\small\begin{verbatim}
%   \usepackage[dvips]{graphicx} ...
%   \includegraphics[width=0.8\linewidth]
%                   {myfile.eps}
%\end{verbatim}
%}
%
%
%%-------------------------------------------------------------------------
%\subsection{Color}
%
%Please refer to the author guidelines on the CVPR 2020 web page for a discussion
%of the use of color in your document.
%
%%------------------------------------------------------------------------
%\section{Final copy}
%
%You must include your signed IEEE copyright release form when you submit
%your finished paper. We MUST have this form before your paper can be
%published in the proceedings.
%
%Please direct any questions to the production editor in charge of these 
%proceedings at the IEEE Computer Society Press: 
%\url{https://www.computer.org/about/contact}. 


{\small
\bibliographystyle{ieee_fullname}
\bibliography{egbib}
}

\end{document}
